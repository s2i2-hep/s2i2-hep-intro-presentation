\begin{frame}
\frametitle{Detector Simulation, Triggering, Event Reconstruction and Visualization} 
\scriptsize{
Challenges surrounding high pile-up simulation,
including the CPU resources needed for large statistics samples
needed to compare with data from high trigger rates, high memory
utilization, generation and handling of the large (min-bias) samples
needed to achieve accurate description of high pile-up collision
events, and a flexible simulation strategy capable of a broad
spectrum of precision in the detector response, from ``fast''
(e.g. parametric) simulation optimized for speed to full simulation
in support of precision measurements and new physics searches
(e.g. in subtle effects on event kinematics due to the presence of
virtual particles at high scale).
Software required to emulate upgraded detectors (including the
trigger system) and support determination of their optimal
configuration and calibration. $\bullet$
Software in support of triggering
during the HL-LHC, including algorithms for the High-level Trigger,
online tracking using GPUs and/or FPGAs, trigger steering, event
building, data ``parking'' (for offline trigger decision), and data
flow control systems. $\bullet$ New approaches to event reconstruction, in
which the processing time depends sensitively on instantaneous
luminosity, including advanced algorithms, vectorization, and
execution concurrency and frameworks that exploit many-core
architectures. In particular, charged particle tracking is expected
to dominate the event processing time under high pile-up
conditions. $\bullet$ Visualization tools, not only in support of upgrade
detector configurations and event displays, but also as a research
tool for data analysis, education, and outreach using modern tools
and technologies for 3D rendering, data and geometry description and
cloud environments.
}
\end{frame}


