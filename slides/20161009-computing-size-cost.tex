\begin{frame}
\frametitle{Software and Computing for the HL-LHC Era }

We don't have precise totals for the cost of global LHC computing 
effort due to the ``in kind'' nature of most of the contributions and different accounting systems.
As a data point we do know that the U.S.\ DOE and NSF jointly invest 
$ \approx \$ 35 $M/year in
ATLAS and CMS software and computing projects, about half in hardware plus operations,
about half in software professionals.
Extrapolating we can conclude that the LHC funding agencies, worldwide, 
likely invest of order $ \$ 100-150 $M/year in these enterprises.

The event rate anticipated for the HL-LHC era is 100 times greater than Run1,
and even assuming the experiments significantly reduce the
amount of data stored per event,
they will be constrained primarily by costs and funding levels,
not by scientific interest.
One long-term goal of a software upgrade for HL-LHC
will be maximizing the return-on-investment to enable break-through
scientific discoveries using the  HL-LHC detectors.


\end{frame}


