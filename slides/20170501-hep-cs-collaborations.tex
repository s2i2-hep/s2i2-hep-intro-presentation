\begin{frame}
\frametitle{Questions from 1st HEP/CS Workshop at NCSA/UIUC}

\footnotesize{
\begin{itemize}
\item What are examples of successful CS-HEP collaborations, and what properties have driven their success?
\item How to align the CS research mechanisms (3 year grants, student developers, conference pubs) with the longer term needs of big science (30 year projects, production software, journal publications)?
\item How to engage a broader slice of the CS community and make scientific computing more “respectable” within CS circles?  (A commonly heard complaint in CS: scientific computing is a ```niche'' research area.)
\item What CS technologies, techniques, and trends could the HEP community adopt, rather than doing everything internally?  (Keeping in mind the long time scales and production needs of HEP.)
\item How could an HEP software institute facilitate interactions between the CS and HEP communities? 
\item What are the incentives for such collaboration for HEP people?  For CS people?  For non-CS people?  E.g. recognition, funding, publications, students, new problems to solve, new places to apply technologies, new solutions to current problems, pride in working on a global-scale problem
\end{itemize}
}

\end{frame}


