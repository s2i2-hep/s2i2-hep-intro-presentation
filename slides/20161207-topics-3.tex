\begin{frame}
\frametitle{ Physics generators, Data Analysis and Interpretation, Data and Software Preservation}
\scriptsize{ 
There are many theory challenges in the HL-LHC era, among them are
improving the precision of SM calculations, better estimation of
systematic uncertainties, and elucidation of promising new physics
signals for the experiments. Software needed to make connection
between observations and theory include matrix element generators,
calculation of higher-order QCD corrections, electroweak
corrections, parton shower modeling, parton matching schemes, and
soft gluon resummation methods. Physics generators that employ
concurrency and exploit many-core architectures will play an
important role in HL-LHC, as well better sharing of code and
processing between LHC experimenters and phenomenologists. $\bullet$ Data
analysis frameworks that include parallelization, optimized event
I/O, data caching, and WAN-based data access. Analysis software
that employs advanced algorithms and efficiently utilizes many-core
architectures. $\bullet$ Tools and technologies for preservation and reuse of
data and software, preservation and re-interpretation of physics
results, analysis providence and workflow ontologies, analysis
capture, and application packaging for platform abstraction. $\bullet$ Future
software repositories and build platforms that leverage advances in
these areas and improved software modularity and quality control
that will allow a broader community of people to effectively
contribute to software in the HL-LHC era.}

\end{frame}


